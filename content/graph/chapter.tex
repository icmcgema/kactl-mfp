\chapter{Graph}

\section{Fundamentals}

\subsection{Como representar}

Em C++, podemos representar uma lista de adjacencias como um vector de vectors, ou um array de vectors:
\begin{lstlisting}
int n, m; cin >> n >> m; // n=vertices, m=arestas
vector<vector<int>> edges(n); // edges[i] = {adjacentes a i}

for (int i = 0; i < m; i++) {
   int u, v; cin >> u >> v;
   edges[u].push_back(v); // u->v
   edges[v].push_back(u); // v->u (para grafos nao direcionados)
}

// tb poderia fazer um dos arrays estaticos:
const int MAXN = 1e5 + 5; // falando que o maximo de N eh 10^5
vector<int> edges[MAXN];
\end{lstlisting}

	\kactlimport{DFS.h}
	\kactlimport{BFS.h}
	\kactlimport{Dijkstra.h}
	\kactlimport{FloydWarshall.h}
	\kactlimport{TopoSort.h}

\section{Network flow}
	% \kactlimport{PushRelabel.h}
	\kactlimport{MinCostMaxFlow.h}
	% \kactlimport{EdmondsKarp.h}
	\kactlimport{Dinic.h}
	\kactlimport{MinCut.h}
	% \kactlimport{GlobalMinCut.h}
	% \kactlimport{GomoryHu.h}

% \section{Matching}
% 	\kactlimport{hopcroftKarp.h}
% 	\kactlimport{DFSMatching.h}
% 	\kactlimport{MinimumVertexCover.h}
% 	\kactlimport{WeightedMatching.h}
% 	\kactlimport{GeneralMatching.h}

\section{DFS algorithms}
	\kactlimport{SCC.h}
	% \kactlimport{BiconnectedComponents.h}
	\kactlimport{2sat.h}
	\kactlimport{EulerWalk.h}

% \section{Coloring}
% 	\kactlimport{EdgeColoring.h}

% \section{Heuristics}
% 	\kactlimport{MaximalCliques.h}
% 	\kactlimport{MaximumClique.h}
% 	\kactlimport{MaximumIndependentSet.h}

% \section{Trees}
	\kactlimport{BinaryLifting.h}
	% \kactlimport{LCA.h}
	% \kactlimport{CompressTree.h}
	% \kactlimport{HLD.h}
	% \kactlimport{LinkCutTree.h}
	% \kactlimport{DirectedMST.h}

% \section{Math}
% 	\subsection{Number of Spanning Trees}
% 		% I.e. matrix-tree theorem.
% 		% Source: https://en.wikipedia.org/wiki/Kirchhoff%27s_theorem
% 		% Test: stress-tests/graph/matrix-tree.cpp
% 		Create an $N\times N$ matrix \texttt{mat}, and for each edge $a \rightarrow b \in G$, do
% 		\texttt{mat[a][b]--, mat[b][b]++} (and \texttt{mat[b][a]--, mat[a][a]++} if $G$ is undirected).
% 		Remove the $i$th row and column and take the determinant; this yields the number of directed spanning trees rooted at $i$
% 		(if $G$ is undirected, remove any row/column).

% 	\subsection{Erdős–Gallai theorem}
% 		% Source: https://en.wikipedia.org/wiki/Erd%C5%91s%E2%80%93Gallai_theorem
% 		% Test: stress-tests/graph/erdos-gallai.cpp
% 		A simple graph with node degrees $d_1 \ge \dots \ge d_n$ exists iff $d_1 + \dots + d_n$ is even and for every $k = 1\dots n$,
% 		\[ \sum _{i=1}^{k}d_{i}\leq k(k-1)+\sum _{i=k+1}^{n}\min(d_{i},k). \]
