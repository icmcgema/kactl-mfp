% Written by Andre Fakhoury, 2021

\chapter{STL}

Possui algoritmos e containers que podem ser utilizados.

\section{Estruturas de dados}

\subsection{vector$<$T$>$}

Um vetor dinâmico, pode ter seu tamanho alterado durante a execucao e tudo mais

\begin{lstlisting}
vector<int> v; // criar um vetor q armazena int
v.push_back(1); // insere no final. O(1)
v.pop_back(); // remove o cara do final. O(1)
v.insert(v.begin(), 1); // insere no comeco. O(n) - CUIDADO COM TLE!!
v.size(); // retorna a qtd de elementos no vetor

vector<long long> x(5, 1); // cria um vetor de 5 elementos valendo 1 cada
for (int i = 0; i < (int) x.size(); i++) cout << x[i]; // imprime 11111
for (int i : x) cout << i; // imprime 11111. Esse for e tipo um "for each"
\end{lstlisting}

\subsection{string}

Um vector de caracteres. Bem melhor do que manipular string em C

\begin{lstlisting}
string x = "aba";
string y = "xxx";
string s = x + y; // concatena, s = abaxxx
cin >> s; // le do input
cout << s; // imprime na tela
x[1] = 'c'; // x = "aca" agora
\end{lstlisting}

\subsection{pair$<$T1, T2$>$}

Um par de elementos quaisquer. Ja possui o comparador de "$<$" implementado, facilitando a comparacao (considera o primeiro e desempata pelo segundo)

\begin{lstlisting}
pair<int, double> p = {1, 3.5};
cout << p.first << " " << p.second; // 1 3.5
p.first = 10;
cout << p.first << " " << p.second; // 10 3.5

// pode ter pair de qqr coisa basicamente
pair<vector<int>, string> p2;
pair<pair<pair<int, int>, int>, int> p3; // tem q pegar p3.first.first.first
\end{lstlisting}
    
\subsection{stack$<$T$>$}

Uma estrutura  de dados de pilha (FILO).

\begin{lstlisting}
stack<int> st;
st.push(1);
st.push(2);
int old = st.top(); // old = 2
st.pop(); // tira da pilha
int cur = st.top(); // cur = 1
\end{lstlisting}

\subsection{queue$<$T$>$}

Uma estrutura  de dados de fila (FIFO).

\begin{lstlisting}
queue<int> qu;
qu.push(1);
qu.push(2);
int old = qu.front(); // old = 1
qu.pop(); // tira da pilha
int cur = qu.front(); // cur = 2
\end{lstlisting}

\subsection{deque$<$T$>$}

Double ended queue: uma fila dupla. Pode inserir e tirar do comeco ou do final.

\begin{lstlisting}
deque<int> dq;
dq.push_back(3); // dq = {3}
dq.push_front(1); // dq = {1, 3}
dq.push_back(2); // dq = {1, 3, 2}
dq.back(); // retorna 2
dq.front(); // retorna 1
dp.pop_back(); // dq = {1, 3}
\end{lstlisting}

\subsection{priority\_queue$<$T$>$}

Uma *max heap*. Faz operacoes de insercao e retornar o topo em O(logn).

\begin{lstlisting}
priority_queue<int> pq;
pq.push(50); // pq = {50}
pq.push(10); // pq = {10, 50}
pq.top(); // retorna 50
pq.pop(); // pq = {10}
\end{lstlisting}

\subsection{set$<$T$>$}

Um conjunto matematico ordenado. Nao armazena valores repetidos. Utiliza uma arvore balanceada. Operacoes em O(logn).

\begin{lstlisting}
set<int> st;
st.insert(10); // st = {10}
st.insert(1); // st = {1, 10}
st.insert(10); // st = {1, 10}
st.erase(10); // st = {1}
cout << *st.begin() << "\n"; // imprime 1

for (int x : st) cout << x << " "; // imprime todos os caras do set
\end{lstlisting}

\subsection{multiset$<$T$>$}

Um conjunto matematico ordenado. **Armazena** valores repetidos. Utiliza uma arvore balanceada. Operacoes em O(logn).

\begin{lstlisting}
multiset<int> ms;
ms.insert(10); // ms = {10}
ms.insert(10); // ms = {10, 10}
ms.insert(1); // ms = {1, 10, 10}
ms.insert(1); // ms = {1, 1, 10, 10}
ms.erase(10); // remove TODOS os 10. ms = {1, 1}
ms.erase(ms.begin()); // remove so o cara do iterador que mandar. ms = {1}
cout << *ms.begin() << "\n"; // imprime 1

for (int x : ms) cout << x << " "; // imprime todos os caras do multiset
\end{lstlisting}

\subsection{map$<$K, V$>$}

Um dicionario "chave, valor". Pra cada chave, tem um valor associado. Caso vc acesse uma chave nao inicializada, e criado uma instância com valor {} (construtor vazio padrao).

\begin{lstlisting}
map<string, int> mp;
mp["aba"] = 10;
mp["axxx"] = 11;
cout << mp["aba"]; // imprime 10
cout << mp["aaaaaaa"]; // imprime 0, inteiro padrao
mp.erase("aba"); // remove a chave "aba" e o valor q tava nela

for (auto p : mp) { // percorre todos os pair<key, value> do map
	cout << p.first << ": " p.second << " ";
} // nesse caso, imprimira "axxx: 11 aaaaaaa: 0 "
\end{lstlisting}

\subsection{unordered}

As estruturas \texttt{unordered\_set$<$T$>$} e \texttt{unordered\_map$<$K, V$>$} sao como os set e map, mas utilizam Hash para fazer as operacoes em O(1). Nao garantem ordem, e podem ser inclusive mais lentos que o set e map (podem ser criados casos de teste que "quebram" o hash, fazendo com que tenha muito conflito).

\section{Iteradores}
    
Um iterador e "como um ponteiro". Sao meio complicados de entender de primeira, mas sao muito utilizados pelos algoritmos da STL. Exemplos:

\begin{lstlisting}
vector<int> v = {1, 2, 3};
vector<int>::iterator it; // tipo do iterador
// se fizer *it, pega o valor do cara daquele endereco

// exemplo
vector<int>::iterador ini = v.begin().
// E um iterador que aponta pro primeiro cara. No caso, *ini = 1

auto ini = v.begin(); // tambem podemos utilizar a tipagem automatica "auto"
// pra nao precisar ficar escrevendo muita coisa

v.end(); // -> um elemento depois do ultimo cara. *v.end() daria erro
v.rbegin(); // iterador reverso (comeca do final pro comeco).
// no caso, v.rbegin() e o ultimo cara (*v.rbegin() = 3)
v.rend(); // iter. reverso, um cara antes do primeiro cara (*v.rend() da erro)
\end{lstlisting}

\section{Algoritmos}

\subsection{sort(begin, end, $<$comparador$>$)}

\ 

\begin{lstlisting}
vector<int> v = {4, 2, 1};
sort(v.begin(), v.end(); // ordena em forma crescente
sort(v.begin(), v.end(), [](int a, int b) {
	return a > b;
}); // utiliza uma funcao lambda para ordenar de forma decrescente

// vc tambem pode fazer com uma funcao sem ser lambda
bool cmp(int a, int b) { return a > b; }
...
sort(v.begin(), v.end(), cmp); 
\end{lstlisting}

\subsection{lower\_bound(begin, end, valor)}

\begin{itemize}[noitemsep]
\item \textbf{lower\_bound:} retorna um iterador para o primeiro elemento \textit{maior ou igual} ao valor enviado.
\item \textbf{upper\_bound:} retorna um iterador para o primeiro elemento \textit{maior} que o valor enviado.
\end{itemize}

\begin{lstlisting}
vector<int> v = {1, 2, 3, 3, 6}; // o vetor deve estar ordenado

auto it1 = lower_bound(v.begin(), v.end(), 1); // it pro indice 0
auto it2 = upper_bound(v.begin(), v.end(), 1); // it pro indice 1
auto it3 = lower_bound(v.begin(), v.end(), 3); // it pro indice 2
auto it4 = upper_bound(v.begin(), v.end(), 3); // it pro indice 4
auto it5 = lower_bound(v.begin(), v.end(), 6); // it pro indice 4
auto it6 = upper_bound(v.begin(), v.end(), 6); // it do v.end()
auto it7 = lower_bound(v.begin(), v.end(), 7); // it do v.end()
auto it8 = lower_bound(v.begin(), v.end(), -1); // it pro indice 0
\end{lstlisting}
    
\subsection{next\_permutation(begin, end)}

Gera permutacoes, muito util para problemas em que precisa testar algo para todas as permutacoes em $O(n!)$

\begin{lstlisting}
vector<int> v = {3, 4, 1};
sort(v.begin(), v.end()); // ordena para ficar na "permutacao inicial"
do {
	for (int i : v) cout << i << " ";
	cout << "\n";
} while(next_permutation(v.begin(), v.end())); // imprime todas as 6 permutacoes do vetor v
\end{lstlisting}

\subsection{min(val1, val2), max(val1, val2)}

Retorna o valor minimo para os valores val1 e val2. Eles devem ser do mesmo tipo, e este tipo deve ter o operador $<$ definido.

\begin{lstlisting}
int a = 4, b = 10;
long long x = 1321, y = 923;

min(a, b); // 4
max(a, b); // 10
min(x, y); // 932
min(a, x); // ERRO, nao sao do mesmo tipo
min( (long long) a, x); // 4, cast antes para long long
min(1LL * a, x); // 4, multiplicacao de long long com INT gera long long
\end{lstlisting}
